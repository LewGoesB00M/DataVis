% Why do we need to submit this? It's not part of the markscheme and you wouldn't
% usually give out your TeX code, just the PDF.

% -------------------------------------------------------------------------------
% Establish page structure & font.
\documentclass[12pt]{report}\usepackage[]{graphicx}\usepackage[]{xcolor}
% maxwidth is the original width if it is less than linewidth
% otherwise use linewidth (to make sure the graphics do not exceed the margin)
\makeatletter
\def\maxwidth{ %
  \ifdim\Gin@nat@width>\linewidth
    \linewidth
  \else
    \Gin@nat@width
  \fi
}
\makeatother

\definecolor{fgcolor}{rgb}{0.345, 0.345, 0.345}
\newcommand{\hlnum}[1]{\textcolor[rgb]{0.686,0.059,0.569}{#1}}%
\newcommand{\hlstr}[1]{\textcolor[rgb]{0.192,0.494,0.8}{#1}}%
\newcommand{\hlcom}[1]{\textcolor[rgb]{0.678,0.584,0.686}{\textit{#1}}}%
\newcommand{\hlopt}[1]{\textcolor[rgb]{0,0,0}{#1}}%
\newcommand{\hlstd}[1]{\textcolor[rgb]{0.345,0.345,0.345}{#1}}%
\newcommand{\hlkwa}[1]{\textcolor[rgb]{0.161,0.373,0.58}{\textbf{#1}}}%
\newcommand{\hlkwb}[1]{\textcolor[rgb]{0.69,0.353,0.396}{#1}}%
\newcommand{\hlkwc}[1]{\textcolor[rgb]{0.333,0.667,0.333}{#1}}%
\newcommand{\hlkwd}[1]{\textcolor[rgb]{0.737,0.353,0.396}{\textbf{#1}}}%
\let\hlipl\hlkwb

\usepackage{framed}
\makeatletter
\newenvironment{kframe}{%
 \def\at@end@of@kframe{}%
 \ifinner\ifhmode%
  \def\at@end@of@kframe{\end{minipage}}%
  \begin{minipage}{\columnwidth}%
 \fi\fi%
 \def\FrameCommand##1{\hskip\@totalleftmargin \hskip-\fboxsep
 \colorbox{shadecolor}{##1}\hskip-\fboxsep
     % There is no \\@totalrightmargin, so:
     \hskip-\linewidth \hskip-\@totalleftmargin \hskip\columnwidth}%
 \MakeFramed {\advance\hsize-\width
   \@totalleftmargin\z@ \linewidth\hsize
   \@setminipage}}%
 {\par\unskip\endMakeFramed%
 \at@end@of@kframe}
\makeatother

\definecolor{shadecolor}{rgb}{.97, .97, .97}
\definecolor{messagecolor}{rgb}{0, 0, 0}
\definecolor{warningcolor}{rgb}{1, 0, 1}
\definecolor{errorcolor}{rgb}{1, 0, 0}
\newenvironment{knitrout}{}{} % an empty environment to be redefined in TeX

\usepackage{alltt}
\usepackage{graphicx} % Required for inserting images
%\usepackage[a4paper, margin=2cm]{geometry}
\graphicspath{{images/}}

\usepackage[utf8]{inputenc}
\usepackage[T1]{fontenc}
\usepackage{float} % Allows for floats to be positioned using [H], which correctly
                   % positions them relative to their location within my LaTeX code.
\usepackage{subcaption}

% -------------------------------------------------------------------------------
% Declare filecontents & biblatex with custom Harvard BCU styling for referencing.
\usepackage[
    useprefix=true,
    maxcitenames=3,
    maxbibnames=99,
    style=authoryear,
    dashed=false, % re-print recurring author names in bibliography
    natbib=true,
    url=false
]{biblatex}

% -------------------------------------------------------------------------------
% Additional styling options to ensure Harvard referencing format.
\renewbibmacro*{volume+number+eid}{%
    \printfield{volume}%
    \setunit*{\addnbspace}
    \printfield{number}%
    \setunit{\addcomma\space}%
    \printfield{eid}}
\DeclareFieldFormat[article]{number}{\mkbibparens{#1}}

\addbibresource{report.bib}

% -------------------------------------------------------------------------------
% Fancy headers; used to show my name, BCU logo and current chapter for the page.
\usepackage{fancyhdr}
\usepackage{calc}
\pagestyle{fancy}

\setlength\headheight{37pt}

\renewcommand{\chaptermark}[1]{%
    \markboth{#1}{}}

\lhead{Lewis Higgins - ID 22133848~~~~~~~~~~~~~~~\includegraphics[width=1.75cm]{bcu logo}}
\fancyhead[R]{\leftmark}

% ------------------------------------------------------------------------------

\title{CMP5352 Report - TITLE NEEDED}
\author{Lewis Higgins - Student ID 22133848}
\date{April 2024}

% -------------------------------------------------------------------------------
\IfFileExists{upquote.sty}{\usepackage{upquote}}{}
\begin{document}

 \pagecolor{yellow} % Change in final

    \makeatletter
    \begin{titlepage}
        \begin{center}
            \includegraphics[width=0.7\linewidth]{bcu logo}\\[4ex]
            {\large \bfseries  \@title }\\[2ex]
            {\large \bfseries  DRAFT VERSION }\\[2ex]
            {\@author}\\[30ex]
            {Word count: XXXX}\\[20ex]
        \end{center}
    \end{titlepage}
    \makeatother
    \thispagestyle{empty}
    \newpage

    \pagecolor{white}

    \begin{abstract}
        text

        text

        text

    \end{abstract}

    \setcounter{page}{0} % Page counter trick so that it actually starts from 1.

    \tableofcontents
    \thispagestyle{empty}

    % Declaring un-numbered chapter because I prefer how it looks.
    \chapter*{Introduction}\label{ch:introduction}
    % Add it to the contents, because un-numbered chapters aren't by default.
    \addcontentsline{toc}{chapter}{Introduction}
    % Put the chapter name in the header.
    \markboth{Introduction}{}

    Text text text

    Text text text

    \pagebreak

    % Declaring un-numbered chapter because I prefer how it looks.
    \chapter*{Motivation and objectives}\label{ch:sec1}
    % Add it to the contents, because un-numbered chapters aren't by default.
    \addcontentsline{toc}{chapter}{Motivation and objectives}
    % Put the chapter name in the header.
    \markboth{Motivation and objectives}{}

\begin{knitrout}
\definecolor{shadecolor}{rgb}{0.969, 0.969, 0.969}\color{fgcolor}\begin{kframe}
\begin{alltt}
\hlkwd{library}\hlstd{(ggplot2)}
\hlkwd{ggplot}\hlstd{(diamonds,} \hlkwd{aes}\hlstd{(}\hlkwc{x} \hlstd{= carat,} \hlkwc{fill} \hlstd{= color))} \hlopt{+}
\hlkwd{geom_histogram}\hlstd{(}\hlkwc{bins} \hlstd{=} \hlnum{30}\hlstd{)}
\end{alltt}
\end{kframe}\begin{figure}[H]

{\centering \includegraphics[width=0.75\linewidth]{figure/unnamed-chunk-1-1} 

}

\caption{\label{fig:fig1}Plot 1}\label{fig:unnamed-chunk-1}
\end{figure}

\end{knitrout}
    \pagebreak
    
\begin{knitrout}
\definecolor{shadecolor}{rgb}{0.969, 0.969, 0.969}\color{fgcolor}\begin{kframe}
\begin{alltt}
\hlkwd{ggplot}\hlstd{(diamonds,} \hlkwd{aes}\hlstd{(}\hlkwc{x} \hlstd{= carat,} \hlkwc{y} \hlstd{= price))} \hlopt{+}
\hlkwd{geom_point}\hlstd{(}\hlkwd{aes}\hlstd{(}\hlkwc{color} \hlstd{= cut))}
\end{alltt}
\end{kframe}\begin{figure}[H]

{\centering \includegraphics[width=0.75\linewidth]{figure/unnamed-chunk-2-1} 

}

\caption{\label{fig:fig2}Plot 1}\label{fig:unnamed-chunk-2}
\end{figure}

\begin{kframe}\begin{alltt}
\hlcom{#geom_smooth()}
\end{alltt}
\end{kframe}
\end{knitrout}

    % Declaring un-numbered chapter because I prefer how it looks.
    \chapter*{Experimental results}\label{ch:sec2}
    % Add it to the contents, because un-numbered chapters aren't by default.
    \addcontentsline{toc}{chapter}{Experimental results}
    % Put the chapter name in the header.
    \markboth{Experimental results}{}

    aaa

    % Declaring un-numbered chapter because I prefer how it looks.
    \chapter*{Summary \& conclusion}\label{ch:sec3}
    % Add it to the contents, because un-numbered chapters aren't by default.
    \addcontentsline{toc}{chapter}{Summary and conclusion}
    % Put the chapter name in the header.
    \markboth{Summary and conclusion}{}

    aaa
