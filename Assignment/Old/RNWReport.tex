% Why do we need to submit this? It's not part of the markscheme and you wouldn't
% usually give out your TeX code, just the PDF.

% -------------------------------------------------------------------------------
% Establish page structure & font.
\documentclass[12pt]{report}\usepackage[]{graphicx}\usepackage[]{xcolor}
% maxwidth is the original width if it is less than linewidth
% otherwise use linewidth (to make sure the graphics do not exceed the margin)
\makeatletter
\def\maxwidth{ %
  \ifdim\Gin@nat@width>\linewidth
    \linewidth
  \else
    \Gin@nat@width
  \fi
}
\makeatother

\definecolor{fgcolor}{rgb}{0.345, 0.345, 0.345}
\newcommand{\hlnum}[1]{\textcolor[rgb]{0.686,0.059,0.569}{#1}}%
\newcommand{\hlstr}[1]{\textcolor[rgb]{0.192,0.494,0.8}{#1}}%
\newcommand{\hlcom}[1]{\textcolor[rgb]{0.678,0.584,0.686}{\textit{#1}}}%
\newcommand{\hlopt}[1]{\textcolor[rgb]{0,0,0}{#1}}%
\newcommand{\hlstd}[1]{\textcolor[rgb]{0.345,0.345,0.345}{#1}}%
\newcommand{\hlkwa}[1]{\textcolor[rgb]{0.161,0.373,0.58}{\textbf{#1}}}%
\newcommand{\hlkwb}[1]{\textcolor[rgb]{0.69,0.353,0.396}{#1}}%
\newcommand{\hlkwc}[1]{\textcolor[rgb]{0.333,0.667,0.333}{#1}}%
\newcommand{\hlkwd}[1]{\textcolor[rgb]{0.737,0.353,0.396}{\textbf{#1}}}%
\let\hlipl\hlkwb

\usepackage{framed}
\makeatletter
\newenvironment{kframe}{%
 \def\at@end@of@kframe{}%
 \ifinner\ifhmode%
  \def\at@end@of@kframe{\end{minipage}}%
  \begin{minipage}{\columnwidth}%
 \fi\fi%
 \def\FrameCommand##1{\hskip\@totalleftmargin \hskip-\fboxsep
 \colorbox{shadecolor}{##1}\hskip-\fboxsep
     % There is no \\@totalrightmargin, so:
     \hskip-\linewidth \hskip-\@totalleftmargin \hskip\columnwidth}%
 \MakeFramed {\advance\hsize-\width
   \@totalleftmargin\z@ \linewidth\hsize
   \@setminipage}}%
 {\par\unskip\endMakeFramed%
 \at@end@of@kframe}
\makeatother

\definecolor{shadecolor}{rgb}{.97, .97, .97}
\definecolor{messagecolor}{rgb}{0, 0, 0}
\definecolor{warningcolor}{rgb}{1, 0, 1}
\definecolor{errorcolor}{rgb}{1, 0, 0}
\newenvironment{knitrout}{}{} % an empty environment to be redefined in TeX

\usepackage{alltt}

\usepackage[a4paper, margin=3cm]{geometry} % Page structure

\usepackage{graphicx} % Required for inserting images
\graphicspath{{images/}} % Any additional images I use (BCU logo, etc) are from here.

\usepackage[utf8]{inputenc} % UTF-8 encoding
\usepackage[T1]{fontenc} % T1 font
\usepackage{float}  % Allows for floats to be positioned using [H], which correctly
                    % positions them relative to their location within my LaTeX code.
\usepackage{subcaption}

% -------------------------------------------------------------------------------
% Declare biblatex with custom Harvard BCU styling for referencing.
\usepackage[
    useprefix=true,
    maxcitenames=3,
    maxbibnames=99,
    style=authoryear,
    dashed=false, 
    natbib=true,
    url=false,
    backend=biber
]{biblatex}

% Additional styling options to ensure Harvard referencing format.
\renewbibmacro*{volume+number+eid}{
    \printfield{volume}
    \setunit*{\addnbspace}
    \printfield{number}
    \setunit{\addcomma\space}
    \printfield{eid}}
\DeclareFieldFormat[article]{number}{\mkbibparens{#1}}

% Declare report.bib as the bibliography source, to be called later via \printbibliography
\addbibresource{report.bib}

% Also import hyperref, used for embedding URLs.
\usepackage{hyperref}

% Set certain colours up for links, whether they link to within the document (citations, contents page)
% or to an external resource (websites)
\hypersetup{
    colorlinks=true,
    linkcolor=black, % Contents page
    urlcolor=blue, % URLs
    citecolor=black, % Citations
}

% -------------------------------------------------------------------------------
% To prevent "Chapter N" display for each chapter
\usepackage[compact]{titlesec}
\usepackage{wasysym}
\usepackage{import}

\titlespacing*{\chapter}{0pt}{-2cm}{0.5cm}
\titleformat{\chapter}[display]
{\normalfont\bfseries}{}{0pt}{\Huge}

% -------------------------------------------------------------------------------
% Custom macro to make an un-numbered footnote.

\newcommand\blfootnote[1]{
    \begingroup
    \renewcommand\thefootnote{}\footnote{#1}
    \addtocounter{footnote}{-1}
    \endgroup
}

% -------------------------------------------------------------------------------
% Fancy headers; used to show my name, BCU logo and current chapter for the page.
\usepackage{fancyhdr}
\usepackage{calc}
\pagestyle{fancy}

\setlength\headheight{37pt} % Set custom header height to fit the image.

\renewcommand{\chaptermark}[1]{%
    \markboth{#1}{}} % Include chapter name.


% Lewis Higgins - ID 22133848           [BCU LOGO]                [CHAPTER NAME]
\lhead{Lewis Higgins - ID 22133848~~~~~~~~~~~~~~~\includegraphics[width=1.75cm]{bcu logo}}
\fancyhead[R]{\leftmark}


% Temp for pagecolor command
\usepackage{xcolor}

\usepackage{listings}

% -------------------------------------------------------------------------------

\title{Visualising Trends in Netflix's Content Library}
\author{Lewis Higgins - Student ID 22133848}
\date{May 2024}

% -------------------------------------------------------------------------------
\IfFileExists{upquote.sty}{\usepackage{upquote}}{}
\begin{document}






 \pagecolor{yellow} % Change in final



    \makeatletter
    \begin{titlepage}
        \begin{center}
            \includegraphics[width=0.7\linewidth]{bcu logo}\\[7ex]
            {\huge \bfseries  \@title }\\[2ex]
            {\large \bfseries  DRAFT VERSION }\\[30ex]
            {\@author}\\[2ex]
            {CMP5352 - Data Visualisation}\\[10ex]
            {Word count: XXXX}\\[10ex]
        \end{center}
    \end{titlepage}
    \makeatother
    \thispagestyle{empty}
    \newpage

  \pagecolor{white} % Change in final

    \begin{abstract}

    As of March 31, 2024, Netflix is the most popular television streaming service
    in the world~\autocite{NetflixSubStats}, with over 269,000,000 active paid memberships. 
    This report aims to analyse the library of content found on Netflix and identify 
    key factors associated with the viewership of this content in the interests of furthering
    profit in the sector.

    \end{abstract} 
    
    % Page counter trick so that the contents page doesn't increment it.
    \setcounter{page}{0}

    \tableofcontents
    \thispagestyle{empty}

    % Declaring un-numbered chapter because I prefer how it looks.
    \chapter*{Introduction}
    % Add it to the contents, because un-numbered chapters aren't by default.
    \addcontentsline{toc}{chapter}{Introduction}
    % Put the chapter name in the header.
    \markboth{Introduction}{}
    
    Data visualisation is a field of data science wherein large datasets are parsed
    using code (most commonly written in Python or R) to produce clear visualisations
    interpretable to a wide audience, even if they do not have in-depth knowledge
    of the dataset. \\

    \noindent This report specifically aims to produce visualisations based on an
    analysis of a large dataset based on the content library available on the Netflix online streaming service. 
    The ever-evolving landscape of streaming services versus traditional television has positioned Netflix
    as a giant in the industry. Central to its success is a vast content library, 
    encompassing a diverse range of movies, TV shows, documentaries, and more. This report delves 
    into a comprehensive dataset of Netflix's content library, aiming to uncover valuable insights 
    and trends. Through exploratory data analysis, the composition of the library will 
    be examined, including factors such as content popularity by genre and release dates.
    This exploration will not only shed light on Netflix's content strategy but also provide 
    potential indicators of current and future trends within the streaming service and the
    broader entertainment market.\\

    

    \noindent This report is split across three sections:
    \begin{itemize}
        \item The \textbf{motivation and objectives} of this report.
        \item The \textbf{results from experiments} on the dataset.
        \item A \textbf{summary} of overall findings.
    \end{itemize}

    \pagebreak

    \chapter{Motivation and objectives}

    % Put the chapter name in the header.
    \markboth{Motivation and objectives}{}
    
    The dominance of Netflix in the entertainment industry is undeniable; as the most popular online streaming
    service with over 269 million active paid memberships \autocite{NetflixSubStats}, it is essential to identify
    what they are doing correctly in the interests of furthering the industry and understanding the preferred
    content of their millions of subscribers.\\

    
    \noindent The dataset in use is \href{https://www.kaggle.com/datasets/shivamb/netflix-shows/data}{sourced from Kaggle}, a 
    public dataset-sharing website. It contains 8807 rows with twelve columns of data:
    
    \begin{itemize}
        \item show\_id - A unique ID for each row of the dataset.
        \item type - "Movie" or "TV Show"
        \item title - The title of the content
        \item director - The director of the content
        \item cast - Actors featured in the content
        \item country - Country of production
        \item date\_added - The date when Netflix added the content to the service
        \item release\_year - The date when the content originally released
        \item rating - The age rating of the content
        \item duration - Duration in minutes (for movies) or seasons (for TV shows)
        \item listed\_in - The genre of the content
        \item description - A large text description of the content.
    \end{itemize}

    % Within this report, data pertaining to the content library of TV shows and movies on Netflix will be analysed
    % to identify key factors in the success of said content. 
    % Netflix is a massive service used by hundreds of millions of people worldwide, and it is therefore
    % important to identify their successes, and how they optimize their content to maximise viewership, revenue 
    % and profit.

    \section{Key questions conerning the data}
    \begin{itemize}
        \item How has the content library grown over time?
        \item Which month of the year has the most releases?
        \item Which \textbf{content type} (movies / TV shows) is more frequent?
        \item Which genres are the most frequent?
        \item What is the age rating distribution of the content library?
    \end{itemize}

    \pagebreak
    
    \chapter{Experimental results}

    % Put the chapter name in the header.
    \markboth{Experimental results}{}

    

\section{Data wrangling}
    To be able to analyse the data, it is essential that the data is first in a state that
    can be analysed effectively. To do so, the data must be explored to identify any potential
    issues, and any that are found must be cleaned.
    
    \subsection{Data exploration}
    A good step to take in the initial exploration of the data is to identify if there are any missing values
    in certain columns. To do so, we can convert any instance of a blank string into R's recognised "NA" type and
    then use is.na() to identify how many there are.

\begin{knitrout}
\definecolor{shadecolor}{rgb}{0.969, 0.969, 0.969}\color{fgcolor}\begin{kframe}
\begin{alltt}
\hlcom{# Convert blank strings to NA.}
\hlstd{dataDf[dataDf} \hlopt{==} \hlstr{""}\hlstd{]} \hlkwb{<-} \hlnum{NA}

\hlcom{# Identify rows containing NA.}
\hlstd{naRows} \hlkwb{<-} \hlstd{dataDf[}\hlkwd{rowSums}\hlstd{(}\hlkwd{is.na}\hlstd{(dataDf)} \hlopt{>} \hlnum{0}\hlstd{),]}
\hlkwd{nrow}\hlstd{(naRows)}
\end{alltt}
\begin{verbatim}
[1] 3475
\end{verbatim}
\begin{alltt}
\hlkwd{nrow}\hlstd{(naRows)}\hlopt{/}\hlkwd{nrow}\hlstd{(dataDf)}\hlopt{*}\hlnum{100}
\end{alltt}
\begin{verbatim}
[1] 39.46
\end{verbatim}
\end{kframe}
\end{knitrout}
    
    We can identify that this dataset has 3475 rows where at least one column has missing data, equating to 
    39.46\% of the dataset. To analyse this in further detail to see specifically which columns contain
    missing data, the naniar package can be used to generate a tibble of where the data is missing.

\begin{knitrout}
\definecolor{shadecolor}{rgb}{0.969, 0.969, 0.969}\color{fgcolor}\begin{table}[H]
\centering
\caption{\label{tab:unnamed-chunk-2}Missing variable summary of the dataset}
\centering
\begin{tabular}[t]{l|r|r}
\hline
variable & n\_miss & pct\_miss\\
\hline
director & 2634 & 29.9\\
\hline
country & 831 & 9.44\\
\hline
cast & 825 & 9.37\\
\hline
date\_added & 10 & 0.114\\
\hline
rating & 4 & 0.0454\\
\hline
duration & 3 & 0.0341\\
\hline
show\_id & 0 & 0\\
\hline
type & 0 & 0\\
\hline
title & 0 & 0\\
\hline
release\_year & 0 & 0\\
\hline
listed\_in & 0 & 0\\
\hline
description & 0 & 0\\
\hline
\end{tabular}
\end{table}

\end{knitrout}

    % unique(dataDf$rating)
    
      
\begin{knitrout}\footnotesize
\definecolor{shadecolor}{rgb}{0.969, 0.969, 0.969}\color{fgcolor}\begin{kframe}
\begin{alltt}
\hlcom{# Factorise categorical columns before the summary for cleaner output.}
\hlstd{factorCols} \hlkwb{<-} \hlkwd{c}\hlstd{(}\hlstr{"type"}\hlstd{,} \hlstr{"country"}\hlstd{,} \hlstr{"release_year"}\hlstd{,} \hlstr{"rating"}\hlstd{,} \hlstr{"duration"}\hlstd{)}
\hlstd{dataDf} \hlkwb{<-} \hlstd{dataDf} \hlopt
    \hlkwd{mutate}\hlstd{(}\hlkwd{across}\hlstd{(}\hlkwd{all_of}\hlstd{(factorCols), as.factor))}

\hlcom{# Convert date_added to a date format using Lubridate.}
\hlstd{dataDf}\hlopt{$}\hlstd{date_added} \hlkwb{<-} \hlkwd{mdy}\hlstd{(dataDf}\hlopt{$}\hlstd{date_added)}


\hlkwd{str}\hlstd{(dataDf,} \hlkwc{vec.len} \hlstd{=} \hlnum{1}\hlstd{,} \hlkwc{width} \hlstd{=} \hlnum{70}\hlstd{,} \hlkwc{strict.width} \hlstd{=} \hlstr{"cut"}\hlstd{)}
\end{alltt}
\begin{verbatim}
'data.frame':	8807 obs. of  12 variables:
 $ show_id     : chr  "s1" ...
 $ type        : Factor w/ 2 levels "Movie","TV Show": 1 2 ...
 $ title       : chr  "Dick Johnson Is Dead" ...
 $ director    : chr  "Kirsten Johnson" ...
 $ cast        : chr  NA ...
 $ country     : Factor w/ 748 levels ", France, Algeria",..: 604 42..
 $ date_added  : Date, format: "2021-09-25" ...
 $ release_year: Factor w/ 74 levels "1925","1942",..: 73 74 ...
 $ rating      : Factor w/ 17 levels "66 min","74 min",..: 8 12 ...
 $ duration    : Factor w/ 220 levels "1 Season","10 min",..: 211 11..
 $ listed_in   : chr  "Documentaries" ...
 $ description : chr  "As her father nears the end of his life, fil"..
\end{verbatim}
\begin{alltt}
\hlcom{# The 'cast' column contains Unicode characters that cause other miscellaneous}
\hlcom{# errors later, so the Unicode characters in each row will be removed by}
\hlcom{# running a gsub on the cast column to remove them.  dataDf$cast <-}
\hlcom{# sapply(dataDf$cast, gsub, pattern = '<U\textbackslash{}\textbackslash{}+200B>', replacement= ' ')}


\hlcom{# summary(dataDf)}
\end{alltt}
\end{kframe}
\end{knitrout}

    \subsection{Data cleaning}

    \section{Visualisations}
\begin{knitrout}
\definecolor{shadecolor}{rgb}{0.969, 0.969, 0.969}\color{fgcolor}

{\centering \includegraphics[width=0.75\linewidth]{C:/Users/Lewis/DataspellProjects/DataVis/Assignment/RNW/figures/unnamed-chunk-3-1} 

}


\end{knitrout}

    \chapter{Summary and conclusion}
    % Add it to the contents, because un-numbered chapters are not by default.
    \addcontentsline{toc}{chapter}{Summary and conclusion}
    % Put the chapter name in the header.
    \markboth{Summary and conclusion}{}

    aaaaa
    
    \printbibliography{}

\end{document}
